\section{Problem Description}\label{sec:prob_descr}

\begin{table}[p]
	\centering
	\caption{Parameters and values.}
	\begin{tabular}{llll}
		\hline
		Symbol & Parameter & Value & Unit \\
		\hline
		$l_a$ & Distance from elevation axis to helicopter body & $0.63$ & \meter \\
		$l_h$ & Distance from pitch axis to motor & $0.18$ & \meter \\
		$K_f$ & Force constant motor & $0.25$ & \newton\per\volt \\
		$J_e$ & Moment of inertia for elevation & $0.83$ & \kilogram\usk\square\meter \\
		$J_t$ & Moment of inertia for travel & $0.83$ & \kilogram\usk\square\meter \\
		$J_p$ & Moment of inertia for pitch & $0.034$ & \kilogram\usk\square\meter \\
		$m_h$ & Mass of helicopter & $1.05$ & \kilogram \\
		$m_w$ & Balance weight & $1.87$ & \kilogram \\
		$m_g$ & Effective mass of the helicopter & $0.05$ & \kilogram \\
		$K_p$ & Force to lift the helicopter from the ground & $0.49$ & \newton \\
		\hline
	\end{tabular}
	\label{tab:parameters}
\end{table}

\begin{table}[p]
	\centering
	\caption{Variables}
	\begin{tabular}{ll}
		\hline
		Symbol & variable \\
		\hline
		$p$ & Pitch \\
		$p_c$ & Pitch setpoint \\
		$\lambda$ & Travel \\
		$\lambda_c$ & Travel rate setpoint \\
		$e$ & Elevation \\
		$e_c$ & Elevation setpoint \\
		$V_f$ & Voltage input, front motor \\
		$V_b$ & Voltage input, back motor \\
		$V_d$ & Voltage difference, $V_f - V_b$ \\
		$V_s$ & Voltage sum, $V_f + V_b$ \\
		$K_{pp}, K_{pd}, K_{ep}, K_{ei}, K_{ed}$ & Controller gains \\
		$T_g$ & Torque exerted by gravity \\
		\hline
	\end{tabular}
	\label{tab:variables}
\end{table}
The lab setup consists of a movable arm equipped with two rotors. The movable arm is hinged to a fixed point, allowing for both lateral and longitudinal motion. The arm is also fitted with a counterweight which effectively slows the dynamics down considerably, as well as lower the amount of rotor thrust needed. The two rotors are fixed to a pitch head assembly hinged to the movable arm. This allows the rotor thrust direction to be indirectly controlled by the differential thrust applied.

From first principles analysis we can derive simple differential equations to describe the system dynamics about the equilibrium, with symbols declared in table \ref{tab:parameters} and \ref{tab:variables}:

\begin{subequations}
	\label{eq:dynamics}
	\begin{align}
		\ddot{p} &= K_1 V_d, \quad K_1 = \frac{K_f l_h}{J_p},\\
		\ddot{\lambda} &= -K_2 p, \quad K_2 = \frac{K_p l_a}{J_t},\\
		\ddot{e} &= K_3 V_s - \frac{T_g}{J_e}, \quad K_3 = \frac{K_f l_a}{J_e},
	\end{align}
\end{subequations}
\begin{equation*}
\end{equation*}

Note simplifications and limitations:
\begin{itemize}
	\item{The time derivative of travel rate is a linear function of pitch only. This small angle approximation does not really hold, as the intended operating range of pitch is as much as $40\degree$.}
	\item{By simple inspection of the lab setup it is clear that the pitch head assembly is hinged slightly above its center of mass. The resulting restoring force, as well as the hinge joint dampening, is not directly included in this model.}
	\item{The rotor thrust is assumed to be proportional to the voltage applied to the motor. This is a simplification. Generally, rotor angular velocity is proportional to the voltage applied, and thrust is proportional to the square of the angular velocity.}
\end{itemize}

To stabilize the plant, adding to \eqref{eq:dynamics} the pitch PD controller
\begin{equation*}
	V_d = K_{pp} (p_c - p) - K_{pd} \dot{p}, \quad K_{pp}, K_{pd} > 0,
\end{equation*}

and the elevation PID controller
\begin{equation*}
	V_s = K_{ei} \int (e_c - e) \ dt +  K_{ep} (e_c - e) - K_{ed} \dot{e}, \quad K_{ei}, K_{ep}, K_{ed} > 0,
\end{equation*}

yields the model equations
\begin{subequations}
\label{eq:model}
\begin{align}
	\ddot{e} + K_{3} K_{ed} \dot{e} + K_{3} K_{ep} e &= K_{3} K_{ep} e_{c}, \label{eq:model_se_al_elev} \\
	\ddot{p} + K_{1} K_{pd} \dot{p} + K_{1} K_{pp} p &= K_{1} K_{pp} p_{c}, \label{eq:model_se_al_pitch} \\
	\ddot{\lambda} &= -K_{2} p, \label{eq:model_se_al_r} 
\end{align}
\end{subequations}
where it is assumed the elevation integral term counteracts the constant disturbance $-\frac{T_g}{J_e}$ and cancel out.

However, in order to achieve a more accurate model we can utilize statistical procedures to best identify the parameters of the first and second order systems. This eliminates errors present in the measurement of the lab setup, and gives us the best model with the given number of states. This gray-box system identification also allows us to verify that the proposed number of states, derived by first principles, yields a model whose performance matches that of the actual system.

